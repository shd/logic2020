\documentclass[10pt,a4paper,oneside]{article}
\usepackage[utf8]{inputenc}
\usepackage[english,russian]{babel}
\usepackage{amsmath}
\usepackage{amsthm}
\usepackage{amssymb}
\usepackage{enumerate}
\usepackage{stmaryrd}
\usepackage{cmll}
\usepackage{mathrsfs}
\usepackage[left=2cm,right=2cm,top=2cm,bottom=2cm,bindingoffset=0cm]{geometry}
\usepackage{proof}
\usepackage{tikz}

\makeatletter
\newcommand{\dotminus}{\mathbin{\text{\@dotminus}}}

\newcommand{\@dotminus}{%
  \ooalign{\hidewidth\raise1ex\hbox{.}\hidewidth\cr$\m@th-$\cr}%
}
\makeatother

\usetikzlibrary{arrows,backgrounds,patterns,matrix,shapes,fit,calc,shadows,plotmarks}
\begin{document}

\begin{center}{\Large\textsc{\textbf{Теоретические (``малые'') домашние задания}}}\\
             \it Математическая логика, ИТМО, М3235-М3239, весна 2020 года\end{center}

\section*{Домашнее задание №1: <<знакомство с исчислением высказываний>>}

\begin{enumerate}

\item Укажите про каждое из следующих высказываний, общезначимо, выполнимо, опровержимо или невыполнимо ли оно:
\begin{enumerate}
\item $\neg A\vee\neg\neg A$
\item $(A\rightarrow\neg B)\vee(B\rightarrow\neg C)\vee(C\rightarrow\neg A)$
\item $A \rightarrow B \vee A$
\item $A \rightarrow B \with B \rightarrow A$
\end{enumerate}

\item Будем говорить, что высказывание $\alpha$ \emph{следует} из высказываний $\gamma_1, \gamma_2, \dots, \gamma_n$
(и будем записывать это как $\gamma_1, \gamma_2, \dots, \gamma_n \models \alpha$), если
при любой оценке, такой, что при всех $i$ выполнено $\llbracket\gamma_i\rrbracket = \text{И}$,
также выполнено и $\llbracket\alpha\rrbracket = \text{И}$.

Пусть даны высказывания $\alpha$ и $\beta$, причём $\alpha\models\beta$, но $\beta\not\models\alpha$. 
Придумайте <<промежуточное>> высказывание $\gamma$, такое, что $\alpha \models \gamma$,
$\gamma \models \beta$, причём $\gamma\not\models\alpha$ и $\beta\not\models\gamma$.

\item Простые высказывания. Докажите высказывания, построив полный вывод:
\begin{enumerate}
\item $\alpha,\beta \vdash \alpha\with\beta$
\item $\alpha,\beta \vdash \alpha\vee\beta$
\item $\neg\alpha,\beta \vdash \alpha\vee\beta$
\item $\alpha,\neg\beta \vdash \alpha\vee\beta$
\item $\gamma \vdash \alpha\to\gamma$
\item $\neg\alpha \vdash \neg\alpha$
\item $\alpha,\beta \vdash \alpha\rightarrow\beta$
\end{enumerate}

\item Ассоциативность и коммутативность.
\begin{enumerate}
\item Докажите или опровергните: $\models \alpha\rightarrow\beta$ влечёт $\models \beta\rightarrow\alpha$.
\item Докажите: $\vdash \alpha\vee\beta \rightarrow \beta\vee\alpha$
\item Докажите: $\vdash \alpha\with\beta \rightarrow \beta\with\alpha$
\end{enumerate}

\item Контрапозиция. Докажите, что $\vdash \alpha\rightarrow\beta \rightarrow \neg (\beta\rightarrow\alpha)$

\item Докажите следующие высказывания, построив полный вывод:
\begin{enumerate}
\item $\neg\alpha,\beta \vdash \neg(\alpha\&\beta)$
\item $\alpha,\neg\beta \vdash \neg(\alpha\&\beta)$
\item $\neg\alpha,\neg\beta \vdash \neg(\alpha\&\beta)$
\item $\neg\alpha,\neg\beta \vdash \neg(\alpha\vee\beta)$
\item $\alpha,\neg\beta \vdash \neg(\alpha\rightarrow\beta)$
\item $\neg\alpha,\beta \vdash \alpha\rightarrow\beta$
\item $\neg\alpha,\neg\beta \vdash \alpha\rightarrow\beta$
\item $\alpha \vdash \neg\neg\alpha$
\end{enumerate}

\end{enumerate}
\end{document}