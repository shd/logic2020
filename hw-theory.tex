\documentclass[10pt,a4paper,oneside]{article}
\usepackage[utf8]{inputenc}
\usepackage[english,russian]{babel}
\usepackage{amsmath}
\usepackage{amsthm}
\usepackage{amssymb}
\usepackage{enumerate}
\usepackage{stmaryrd}
\usepackage{cmll}
\usepackage{mathrsfs}
\usepackage[left=2cm,right=2cm,top=2cm,bottom=2cm,bindingoffset=0cm]{geometry}
\usepackage{proof}
\usepackage{tikz}
\usepackage{multicol}

\makeatletter
\newcommand{\dotminus}{\mathbin{\text{\@dotminus}}}

\newcommand{\@dotminus}{%
  \ooalign{\hidewidth\raise1ex\hbox{.}\hidewidth\cr$\m@th-$\cr}%
}
\makeatother

\usetikzlibrary{arrows,backgrounds,patterns,matrix,shapes,fit,calc,shadows,plotmarks}
\begin{document}

\begin{center}{\Large\textsc{\textbf{Теоретические (``малые'') домашние задания}}}\\
             \it Математическая логика, ИТМО, М3235-М3239, весна 2020 года\end{center}

\section*{Домашнее задание №1: <<знакомство с исчислением высказываний>>}

\begin{enumerate}

\item Укажите про каждое из следующих высказываний, общезначимо, выполнимо, опровержимо или невыполнимо ли оно:
\begin{enumerate}
\item $\neg A\vee\neg\neg A$
\item $(A\rightarrow\neg B)\vee(B\rightarrow\neg C)\vee(C\rightarrow\neg A)$
\item $A \rightarrow B \vee A$
\item $A \rightarrow B \with B \rightarrow A$
\item $A \rightarrow B \rightarrow \neg B\rightarrow\neg A$.
\end{enumerate}

\item Будем говорить, что высказывание $\alpha$ \emph{следует} из высказываний $\gamma_1, \gamma_2, \dots, \gamma_n$
(и будем записывать это как $\gamma_1, \gamma_2, \dots, \gamma_n \models \alpha$), если
при любой оценке, такой, что при всех $i$ выполнено $\llbracket\gamma_i\rrbracket = \text{И}$,
также выполнено и $\llbracket\alpha\rrbracket = \text{И}$.

Пусть даны высказывания $\alpha$ и $\beta$, причём $\alpha\models\beta$, но $\beta\not\models\alpha$. 
Придумайте <<промежуточное>> высказывание $\gamma$, такое, что $\alpha \models \gamma$,
$\gamma \models \beta$, причём $\gamma\not\models\alpha$ и $\beta\not\models\gamma$.

\item Простые высказывания. Докажите высказывания, построив полный вывод:
\begin{enumerate}
\item $\alpha,\beta \vdash \alpha\with\beta$
\item $\alpha,\beta \vdash \alpha\vee\beta$
\item $\neg\alpha,\beta \vdash \alpha\vee\beta$
\item $\alpha,\neg\beta \vdash \alpha\vee\beta$
\item $\gamma \vdash \alpha\to\gamma$
\item $\neg\alpha \vdash \neg\alpha$
\item $\alpha,\beta \vdash \alpha\rightarrow\beta$
\end{enumerate}

\item Ассоциативность и коммутативность.
\begin{enumerate}
\item Докажите или опровергните: $\models \alpha\rightarrow\beta$ влечёт $\models \beta\rightarrow\alpha$.
\item Докажите: $\vdash \alpha\vee\beta \rightarrow \beta\vee\alpha$
\item Докажите: $\vdash \alpha\with\beta \rightarrow \beta\with\alpha$
\end{enumerate}

\item Контрапозиция. $\vdash (\alpha\rightarrow\beta) \rightarrow \neg\beta\rightarrow\neg\alpha$.

\item Докажите следующие высказывания, построив полный вывод:
\begin{enumerate}
\item $\neg\alpha,\beta \vdash \neg(\alpha\&\beta)$
\item $\alpha,\neg\beta \vdash \neg(\alpha\&\beta)$
\item $\neg\alpha,\neg\beta \vdash \neg(\alpha\&\beta)$
\item $\neg\alpha,\neg\beta \vdash \neg(\alpha\vee\beta)$
\item $\alpha,\neg\beta \vdash \neg(\alpha\rightarrow\beta)$
\item $\neg\alpha,\beta \vdash \alpha\rightarrow\beta$
\item $\neg\alpha,\neg\beta \vdash \alpha\rightarrow\beta$
\item $\alpha \vdash \neg\neg\alpha$
\end{enumerate}

\end{enumerate}

\section*{Домашнее задание №2: <<интуиционистское исчисление высказываний>>}

\begin{enumerate}
\item Долги по теореме о полноте ИВ. Докажите:
\begin{enumerate}
\item $\vdash \alpha\vee\neg\alpha$
\item $\Gamma,\alpha \vdash \phi$ и $\Gamma,\neg\alpha\vdash\phi$ влечёт $\Gamma\vdash\phi$
\end{enumerate}

\item Постройте дерево вывода для следующих высказываний интуиционистской логики
(в данных примерах $\neg\alpha$ --- сокращение для $\alpha\rightarrow\bot$):

\begin{enumerate}
\item $\vdash\alpha\rightarrow\alpha$
\item $\alpha\rightarrow\beta\vdash\neg\beta\rightarrow\neg\alpha$
\item $\vdash\alpha\vee\beta\rightarrow\beta\vee\alpha$
\item $\vdash\alpha\with\beta\rightarrow\beta\with\alpha$
\item $\vdash\neg\neg(\alpha\vee\neg\alpha)$
\end{enumerate}

\item Постройте примеры частично упорядоченных множеств:
\begin{enumerate}
\item определено $a + b$, но не $a \cdot b$;
\item определено $a \cdot b$, но не $a + b$;
\item является решёткой, но не является дистрибутивной решёткой;
\item является дистрибутивной, но не импликативной решёткой;
\item является импликативной, но не имеет нуля;
\end{enumerate}

\item Решётки. Покажите, что следующие утверждения выполнены в любой решётке и
при любых $a$, $b$ и $c$:

\begin{enumerate}
\item Коммутативность: $a \cdot b = b \cdot a$ и $a + b = b + a$
\item Ассоциативность: $a \cdot (b \cdot c) = (a \cdot b) \cdot c$ и $a + (b + c) = (a + b) + c$
\item Законы поглощения: $a \cdot (a + b) = a$ и $a + (a \cdot b) = a$
\item Верно ли, что если $a \sqsubseteq b$, то $a + c \sqsubseteq b + c$ и $a \cdot c \sqsubseteq b \cdot c$?
\item Верно ли, что если $a + c \sqsubseteq b + c$ или $a \cdot c \sqsubseteq b \cdot c$, то $a \sqsubseteq b$?
\end{enumerate}

\item В любой дистрибутивной решётке
\begin{enumerate}
\item $(a \cdot b) + c = (a + c) \cdot (b + c)$.
\item Нет \emph{диамантов}: таких пяти элементов $p$, $q$, $r$, $s$, $t$, что 
$p \sqsubseteq q, r, s \sqsubseteq t$, и при этом $q$, $r$ и $s$ --- несравнимы.

\begin{center}\tikz{
\node at (1,2)   (A) {t};
\node at (0,0.5) (B) {q};
\node at (1,0.5) (C) {r};
\node at (2,0.5) (D) {s};
\node at (1,-1)  (E) {p};
\draw[->] (A) to (B); \draw[->] (B) to (E);
\draw[->] (A) to (C); \draw[->] (C) to (E);
\draw[->] (A) to (D); \draw[->] (D) to (E);
}
\end{center}

При этом, если на данной диаграмме выполнено какое-то вычисление
(например, $q + r = t$), то оно должно быть выполнено и в исходной дистрибутивной решётке.


\item Нет \emph{пентагонов}: таких пяти элементов $p$, $q$, $r$, $s$, $t$, что 
$p \sqsubseteq q, r, s \sqsubseteq t$, также $r \sqsubseteq s$, элемент же
$q$ не сравним с $r$ и $s$.

\begin{center}\tikz{
\node at (5,2)   (A1) {t};
\node at (4,0.5) (B1) {q};
\node at (6,1)   (C1) {s};
\node at (6,0)   (D1) {r};
\node at (5,-1)  (E1) {p};
\draw[->] (A1) to (B1); \draw[->] (B1) to (E1);
\draw[->] (A1) to (C1); \draw[->] (C1) to (D1); \draw[->] (D1) to (E1);
}\end{center}

При этом, если на данной диаграмме выполнено какое-то вычисление
(например, $q + r = t$), то оно должно быть выполнено и в исходной дистрибутивной решётке.

\end{enumerate}

\item Покажите, что в импликативной решётке
\begin{enumerate}
\item выполнена дистрибутивность;
\item Из $a \sqsubseteq b$ следует $b\to c \sqsubseteq a\to c$ и $c\to a \sqsubseteq c \to b$;
\item Из $a \sqsubseteq b \to c$ следует $a \cdot b \sqsubseteq c$;
\item $a \sqsubseteq b$ выполнено тогда и только тогда, когда $a \to b = 1$;
\item $b \sqsubseteq a \rightarrow b$;
\item $a \rightarrow b \sqsubseteq ((a \rightarrow (b \rightarrow c)) \rightarrow (a \rightarrow c))$;
\item $a \sqsubseteq b \rightarrow a \cdot b$;
\item $a \rightarrow c \sqsubseteq (b \rightarrow c) \rightarrow (a + b \rightarrow c)$
\end{enumerate}

\item Рассмотрим топологию $\langle X, \Omega\rangle$ (напомним, что здесь $\Omega$ --- множество
всех открытых подмножеств множества $X$). Рассмотрим множество $\Omega$, частично упорядоченное
отношением <<быть подмножеством>>. Покажите, что получившаяся конструкция:

\begin{enumerate}
\item решётка;
\item дистрибутивная решётка;
\item импликативная решётка;
\item псевдобулева алгебра;
\item не является булевой алгеброй.
\end{enumerate}

\item Покажите, что булева алгебра --- булева алгебра.
\item Покажите, что подмножества некоторого множества, упорядоченнные отношением 
<<быть подмножеством>> --- булева алгебра.

\item Покажите недоказуемость следующих высказываний интуиционистской логики, построив 
\emph{конечные} псевдобулевы алгебры (т.е. частично упорядоченные множества с конечным
количеством элементов), в которых следующие высказывания не истинны:

\begin{enumerate}
\item $A \vee \neg A$
\item $(((A \rightarrow B) \rightarrow A) \rightarrow A)$
\item $(A \rightarrow (B \vee \neg B)) \vee (\neg A \rightarrow (B \vee \neg B))$
\item $\neg A \vee \neg\neg A $
\end{enumerate}

\item Теорема о полноте алгебр Гейтинга как моделей для интуиционистского исчисления высказываний.

Уточним определения, данные на лекции:
\begin{enumerate}
\item Будем писать $\alpha\sqsubseteq\beta$, если $\alpha\vdash\beta$.
\item Будем писать $\alpha\approx\beta$, если имеет место $\alpha\sqsubseteq\beta$ и $\beta\sqsubseteq\alpha$.
\item Будем писать $[\alpha]$ для класса эквивалентности, порождённого по формуле $\alpha$:
$[\alpha] = \{\phi\ |\ \phi \approx \alpha \}$
\end{enumerate}
Интуиция здесь такая: высказывание тем ближе к 0 (к лжи), чем меньше ситуаций, в которых оно истинно. 
Поэтому если $\alpha\vdash\beta$, то $\alpha$ не больше $\beta$: возможно, $\beta$ истинно ещё в каких-то
ситуациях, в которых ложно $\alpha$ (но не наоборот).

Тогда докажите следующие утверждения:

\begin{enumerate}
\item $(\approx)$ есть действительно отношение эквивалентности.
\item $[\alpha\with\beta]$ --- наибольшая нижняя грань $[\alpha]$ и $[\beta]$ в алгебре Линденбаума.
То есть, $\alpha\with\beta \sqsubseteq \alpha$, $\alpha\with\beta \sqsubseteq \beta$, и из $\tau \sqsubseteq \alpha$
и $\tau \sqsubseteq \beta$ следует $\tau \sqsubseteq \alpha\with\beta$. Также поясните, почему 
нам достаточно доказать эти утверждения для отдельных представителей,
чтобы доказать свойства для классов эквивалентности.
\item $[\alpha\vee\beta]$ --- наименьшая верхняя грань $[\alpha]$ и $[\beta]$.
\item $[\alpha\rightarrow\beta]$ --- псевдополнение $[\alpha]\rightarrow[\beta]$.
\item $[\bot]$ --- ноль.
\item $[\neg\alpha]$ --- псевдодополнение до нуля $\sim[\alpha]$.
\end{enumerate}

\end{enumerate}

\section*{Домашнее задание №3: <<корректность и дизъюнктивность интуиционистского исчисления высказываний>>}
\begin{enumerate}

\item Теорема о корректности: если $\vdash \alpha$, то $\models \alpha$ в любой алгебре Гейтинга.
Поскольу в принятом нами интуиционистском исчислении высказываний доказываются не высказывания, а
некоторые сложные условные выражения (записи вида $\Gamma \vdash \alpha$), то и оценкой для
данных выражений мы выберем неравенство. А именно, 
если $\Gamma = \{\gamma_1, \dots, \gamma_n\}$, $g_i = \llbracket \gamma_i \rrbracket$ и
$a = \llbracket \alpha \rrbracket$, то выражению $\Gamma \vdash \alpha$ мы сопоставим неравенство
$$g_1 \cdot g_2 \cdot \dots \cdot g_n \sqsubseteq a$$
и записывать его будем как $\Gamma \sqsubseteq a$.
Также, на случай $G = \varnothing$ положим, что $\varnothing \sqsubseteq a$ означает $a = 1$
(это естественно предположить, поскольку к любому $G$ всегда можно добавить некоторое 
$g_0 = 1$, при этом смысл выражения $g_0 \cdot g_1 \cdot g_2 \cdot \dots \cdot g_n \sqsubseteq a$ 
останется прежним).

Отметим, что следующие утвержедения очевидны (или уже показаны ранее):
\begin{itemize}
\item \emph{Аксиома.} $G,a \sqsubseteq a$.
\item \emph{Введение $(\with)$.} Если $G \sqsubseteq b$ и $G \sqsubseteq c$, то $G \sqsubseteq b \cdot c$.
\item \emph{Удаление $(\with)$.} Если $G \sqsubseteq a \cdot b$, то $G \sqsubseteq a$ и $G \sqsubseteq b$.
\item \emph{Введение $(\vee)$.} Если $G \sqsubseteq b$, то $G \sqsubseteq b + c$ и $G \sqsubseteq c + b$.
\end{itemize}

Теперь осталось заполнить промежутки и получить полноценное доказательство.
А именно, покажите, что:

\begin{enumerate}
\item \emph{Введение $(\to)$.} Если $G,a \sqsubseteq b$, то $G \sqsubseteq a \to b$.
Убедитесь, что если $a \sqsubseteq b$, то $\varnothing \sqsubseteq a \to b$.
\item \emph{Удаление $(\to)$.} Если $G \sqsubseteq a \to b$ и $G \sqsubseteq a$, то $G \sqsubseteq b$.
\item \emph{Удаление $(\vee)$.} Если $G,a \sqsubseteq c$, $G,b \sqsubseteq c$ и $G \sqsubseteq a+b$,
то $G \sqsubseteq c$.
\item \emph{Удаление лжи.} Если $G \sqsubseteq 0$, то $G \sqsubseteq b$ при любом $b$.
\item Основываясь на доказанных выше утверждениях, покажите теорему о корректности в целом.
\end{enumerate}

\item Поясните, как соотносится $G \sqsubseteq a$ со следующими системой (подзадача a) 
и совокупностью (подзадача b) неравенств:

\begin{multicols}{2}%
(a) $$\left\{\begin{array}{l}
g_1 \sqsubseteq a\\
g_2 \sqsubseteq a\\
\dots\\
g_n \sqsubseteq a\end{array}\right.
$$

(b) $$\left[\begin{array}{l}
g_1 \sqsubseteq a\\
g_2 \sqsubseteq a\\
\dots\\
g_n \sqsubseteq a\end{array}\right.$$
\end{multicols}
То есть, следует ли какое-нибудь утверждение из какого-нибудь другого, и если да, то докажите,
если нет --- предложите контрпример.

\item Покажите, что $\Gamma(\mathcal{A})$ --- алгебра Гейтинга, если $\mathcal{A}$ --- алгебра Гейтинга.

\item Покажите или опровергните, что если $\Gamma(\mathcal{B})$~--- алгебра Гейтинга, 
то и $\mathcal{B}$ --- алгебра Гейтинга.

\item Покажите, что отображение $\varphi: \Gamma(\mathcal{A})\to\mathcal{A}$, определённое как:
$$\varphi(g) = \left\{\begin{array}{ll}
   g,& \mbox{если }g \sqsubset \omega\\
   1_\mathcal{A},& \mbox{если }g = \omega \mbox{ или } g = 1_{\Gamma(\mathcal{A})}\end{array}\right.$$
действительно является гомоморфизмом.

\item Является ли требование на сохранение нуля в определении гомоморфизма алгебр Гейтинга обязательным?

Если точнее, пусть $\varphi: \mathcal{A} \to \mathcal{B}$ --- отображение алгебр Гейтинга, 
сохраняющее операции: $\varphi(a\star b) = \varphi(a)\star\varphi(b)$ и $\varphi(\neg a) = \neg \varphi (a)$.
Всегда ли $\varphi(0_\mathcal{A}) = 0_\mathcal{B}$ и $\varphi$ --- гомоморфизм.

\item Покажем, что если $\alpha$ доказано в интуиционистском исчислении высказываний 
в стиле Гильберта (с изменённой 10 аксиомой:
$\alpha\to\neg\alpha\to\beta$), то оно может быть доказано и в интуиционистской. 

Для этого покажите, что:
\begin{enumerate}
\item Все аксиомы 1-9 выполнены (подзадачи a.1 --- a.9): если $\alpha$ --- аксиома, то
$\vdash_\text{и}\alpha$.
\item $\vdash_\text{и}\alpha\rightarrow\neg\alpha\rightarrow\beta$.
\item Покажите правило Modus Ponens: если $\vdash_\text{и} \alpha$ и $\vdash_\text{и} \alpha\to\beta$,
то $\vdash_\text{и} \beta$.
\end{enumerate}

\item Как доказать, что если $\vdash_\text{и}\alpha$, то $\vdash_\text{к}\alpha$?
Придумайте схему доказательства, для доказательства отдельных утверждений можно пользоваться 
теоремой о полноте К.И.В.

\item Как доказать, что если $\vdash_\text{и}\alpha$, то $\vdash\alpha$ в интуиционистском исчислении высказываний
в стиле Гильберта? Придумайте схему доказательства.

\item Покажем теорему Гливенко. Для этого покажем следующее:

\begin{enumerate}
\item Если $\vdash_\text{и}\alpha$, то $\vdash_\text{и}\neg\neg\alpha$.
\item $\vdash_\text{и}\neg\neg(\neg\neg\alpha\to\alpha)$.
\item Вспользовавшись предыдущими пунктами и задачами, докажите теорему Гливенко.
\end{enumerate}
\end{enumerate}

\section*{Домашнее задание №4: <<Исчисление предикатов>>}
\begin{enumerate}
\item Докажите следующие формулы в исчислении предикатов:
\begin{enumerate}
\item $\forall x.\phi\rightarrow \phi$
\item $(\forall x.\phi)\rightarrow (\exists x.\phi)$
\item $(\forall x.\forall x.\phi) \rightarrow (\forall x.\phi)$
\item $(\forall x.\phi) \rightarrow (\neg \exists x.\neg \phi)$ 
\item $(\exists x.\phi) \rightarrow (\neg \forall x.\neg \phi)$
\item $(\forall x.\neg\phi) \rightarrow (\neg \exists x.\phi)$ 
\item $(\exists x.\neg\phi) \rightarrow (\neg \forall x.\phi)$
\end{enumerate}

\item Опровергните формулы $\phi\rightarrow\forall x. \phi$ и $(\exists x.\phi)\rightarrow (\forall x.\phi)$

\item Все правила и аксиомы с кванторами имеют дополнительные ограничения
на свободу переменных (свободу для подстановки). Для каждого из правил и 
каждой из аксиом найдите по примеру, когда эти ограничения существенны 
(они запрещают доказательства, выводящие опровержимые формулы).

\item Рассмотрим формулу $\alpha$ с двумя свободными переменными $x$ и $y$ (мы предполагаем,
что эти метапеременные соответствуют разным переменным).
Определите, какие из сочетаний кванторов выводятся из каких --- и приведите соответствующие
доказательства или опровержения:
\begin{enumerate}
\item $\forall x.\forall y.\alpha$, $\forall y.\forall x.\alpha$
\item $\exists x.\exists y.\alpha$, $\exists y.\exists x.\alpha$
\item $\forall x.\forall y.\alpha$, $\forall x.\exists y.\alpha$, $\exists x.\forall y.\alpha$, $\exists x.\exists y.\alpha$
\item $\forall x.\exists y.\alpha$, $\exists y.\forall x.\alpha$
\end{enumerate}

\item Научимся выносить квантор всеобщности <<наружу>>:
\begin{enumerate}
\item Покажите, что если $x$ не входит свободно в $\alpha$, то
$$
\vdash(\alpha \vee \forall x.\beta) \rightarrow (\forall x.\alpha\vee\beta)\quad
\mbox{и}\quad
\vdash((\forall x.\beta)\vee\alpha) \rightarrow (\forall x.\beta\vee\alpha)\quad
$$
\item Покажите, что $$\vdash((\forall x.\alpha) \vee (\forall y.\beta)) \rightarrow \forall p.\forall q.\alpha[x:=p]\vee\beta[y := q]$$
где $p$ и $q$ --- свежие переменные, не входящие в формулу. Заметим, что в частном случае $x$ может совпадать с $y$.
\item Докажите аналогичные утверждения для $\with$.
\item Как будут сформулированы аналогичные утверждения для $\rightarrow$ и $\neg$? Сформулируйте и докажите их.
\end{enumerate}

\item Научимся вносить квантор всеобщности <<внутрь>>:
\begin{enumerate}
\item Покажите, что если $x$ не входит свободно в $\alpha$, то
$$
\vdash (\forall x.\alpha\vee\beta)\rightarrow(\alpha \vee \forall x.\beta)\quad
\mbox{и}\quad
\vdash (\forall x.\beta\vee\alpha)\rightarrow((\forall x.\beta)\vee\alpha)\quad
$$

\item Покажите, что если $p$ не входит свободно в $\beta$ и $q$ не входит свободно в $\alpha$, то
$$\vdash(\forall p.\forall q.\alpha\vee\beta) \rightarrow (\forall x.\alpha[p := x]) \vee (\forall y.\beta[q := y])$$
при условии, что $x$ свободно для подстановки вместо $p$ в $\alpha$ и $y$ свободно для подстановки вместо $q$ в $\beta$.
\item Докажите аналогичные утверждения для $\with$.
\item Как будут сформулированы аналогичные утверждения для $\rightarrow$ и $\neg$? Сформулируйте и докажите их.
\end{enumerate}

\item Сформулируйте и докажите аналогичные предыдущим пунктам утверждения для квантора существования.

\item Научимся работать со спрятанными глубоко кванторами. Пусть $\vdash\alpha\rightarrow\beta$, тогда:
\begin{enumerate}
\item Докажите: $$\vdash\psi\vee\alpha \rightarrow \psi\vee\beta\quad\vdash\psi\with\alpha \rightarrow \psi\with\beta\quad
\vdash(\psi\rightarrow\alpha) \rightarrow (\psi\rightarrow\beta)\quad\vdash(\beta\rightarrow\psi) \rightarrow (\alpha\rightarrow\psi)$$
\item Сформулируйте и докажите аналогичное свойство для отрицания.
\item Докажите $\vdash(\forall x.\alpha)\rightarrow(\forall x.\beta)$. 
Надо ли наложить на формулы $\alpha$ и $\beta$ какие-либо ограничения?
\item Докажите $\vdash(\exists x.\alpha)\rightarrow(\exists x.\beta)$. 
Надо ли наложить на формулы $\alpha$ и $\beta$ какие-либо ограничения?
\end{enumerate}

\item Формулой исчисления предикатов с \emph{поверхностными} кванторами (формулой в предварённой форме) назовём формулу,
соответствующую нетерминалу $\psi$ в грамматике

$$\psi ::= \forall x.\psi | \exists x.\psi | \sigma$$

где $\sigma$ --- это формула, не содержащая кванторов. Иными словами, это формула, в которой все кванторы снаружи ---
квантор не может быть указан внутри конъюнкции, дизъюнкции, импликации или отрицания.

Опираясь на доказанные выше леммы, докажите, что
если $\alpha$ --- формула, то для неё найдётся такая формула $\beta$ с поверхностными кванторами, что:
\begin{enumerate}
\item $\vdash\alpha\rightarrow\beta$
\item $\vdash\beta\rightarrow\alpha$
\end{enumerate}


\end{enumerate}

\end{document}
