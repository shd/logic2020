\documentclass[11pt,a4paper,oneside]{scrartcl}
\usepackage[utf8]{inputenc}
\usepackage[english,russian]{babel}
\usepackage[top=1cm,bottom=1cm,left=1cm,right=1cm]{geometry}

\begin{document}
\pagestyle{empty}

\begin{center}
{\large\scshape\bfseries Список вопросов к курсу <<Математическая логика>>}\\
ИТМО, группы M3234--M3239, весна 2020 г.
\end{center}

%\vspace{0.3cm}

\begin{enumerate}
\item Исчисление высказываний. Общезначимость, следование, доказуемость, выводимость. Корректность, полнота, непротиворечивость.
Теорема о дедукции для исчисления высказываний.
\item Теорема о полноте исчисления высказываний.
\item Интуиционистское исчисление высказываний. BHK-интерпретация. Решётки. Булевы и псевдобулевы алгебры.
\item Алгебра Линденбаума. Полнота интуиционистского исчисления высказываний в псевдобулевых алгебрах.
\item Модели Крипке. Сведение моделей Крипке к псевдобулевым алгебрам. Нетабличность интуиционистского исчисления высказываний.
\item Гёделева алгебра. Операция $\Gamma(A)$. Дизъюнктивность интуиционистского исчисления высказываний.
\item Исчисление предикатов. Общезначимость, следование, выводимость. Теорема о дедукции в исчислении предикатов.
\item Непротиворечивые множества формул. Доказательство существования моделей у непротиворечивых множеств формул 
в бескванторном исчислении предикатов.
\item Теорема Гёделя о полноте исчисления предикатов. Доказательство полноты исчисления предикатов.
\item Теории первого порядка, структуры и модели. Аксиоматика Пеано. Арифметические операции. Формальная арифметика. 
\item Примитивно-рекурсивные и рекурсивные функции. Функция Аккермана. Примитивная рекурсивность 
арифметических функций, функций вычисления простых чисел, частичного логарифма.
\item Выразимость отношений и представимость функций в формальной арифметике. 
Представимость примитивов $N$, $Z$, $S$, $U$ в формальной арифметике.
\item Бета-функция Гёделя. Представимость примитивов $R$ и $M$ и рекурсивных функций в формальной арифметике.
\item Гёделева нумерация. Рекурсивность представимых в формальной арифметике функций.
\item Непротиворечивость и $\omega$-непротиворечивость. Первая теорема Гёделя о неполноте арифметики, её неформальный смысл.
\item Формулировка первой теоремы Гёделя о неполноте арифметики в форме Россера, её неформальный смысл. 
Формулировка второй теоремы Гёделя о неполноте арифметики, $Consis$. 
Неформальное пояснение метода доказательства.
\end{enumerate}

\end{document}